% Options for packages loaded elsewhere
\PassOptionsToPackage{unicode}{hyperref}
\PassOptionsToPackage{hyphens}{url}
%
\documentclass[
]{article}
\usepackage{amsmath,amssymb}
\usepackage{iftex}
\ifPDFTeX
  \usepackage[T1]{fontenc}
  \usepackage[utf8]{inputenc}
  \usepackage{textcomp} % provide euro and other symbols
\else % if luatex or xetex
  \usepackage{unicode-math} % this also loads fontspec
  \defaultfontfeatures{Scale=MatchLowercase}
  \defaultfontfeatures[\rmfamily]{Ligatures=TeX,Scale=1}
\fi
\usepackage{lmodern}
\ifPDFTeX\else
  % xetex/luatex font selection
\fi
% Use upquote if available, for straight quotes in verbatim environments
\IfFileExists{upquote.sty}{\usepackage{upquote}}{}
\IfFileExists{microtype.sty}{% use microtype if available
  \usepackage[]{microtype}
  \UseMicrotypeSet[protrusion]{basicmath} % disable protrusion for tt fonts
}{}
\makeatletter
\@ifundefined{KOMAClassName}{% if non-KOMA class
  \IfFileExists{parskip.sty}{%
    \usepackage{parskip}
  }{% else
    \setlength{\parindent}{0pt}
    \setlength{\parskip}{6pt plus 2pt minus 1pt}}
}{% if KOMA class
  \KOMAoptions{parskip=half}}
\makeatother
\usepackage{xcolor}
\usepackage[margin=1in]{geometry}
\usepackage{color}
\usepackage{fancyvrb}
\newcommand{\VerbBar}{|}
\newcommand{\VERB}{\Verb[commandchars=\\\{\}]}
\DefineVerbatimEnvironment{Highlighting}{Verbatim}{commandchars=\\\{\}}
% Add ',fontsize=\small' for more characters per line
\usepackage{framed}
\definecolor{shadecolor}{RGB}{248,248,248}
\newenvironment{Shaded}{\begin{snugshade}}{\end{snugshade}}
\newcommand{\AlertTok}[1]{\textcolor[rgb]{0.94,0.16,0.16}{#1}}
\newcommand{\AnnotationTok}[1]{\textcolor[rgb]{0.56,0.35,0.01}{\textbf{\textit{#1}}}}
\newcommand{\AttributeTok}[1]{\textcolor[rgb]{0.13,0.29,0.53}{#1}}
\newcommand{\BaseNTok}[1]{\textcolor[rgb]{0.00,0.00,0.81}{#1}}
\newcommand{\BuiltInTok}[1]{#1}
\newcommand{\CharTok}[1]{\textcolor[rgb]{0.31,0.60,0.02}{#1}}
\newcommand{\CommentTok}[1]{\textcolor[rgb]{0.56,0.35,0.01}{\textit{#1}}}
\newcommand{\CommentVarTok}[1]{\textcolor[rgb]{0.56,0.35,0.01}{\textbf{\textit{#1}}}}
\newcommand{\ConstantTok}[1]{\textcolor[rgb]{0.56,0.35,0.01}{#1}}
\newcommand{\ControlFlowTok}[1]{\textcolor[rgb]{0.13,0.29,0.53}{\textbf{#1}}}
\newcommand{\DataTypeTok}[1]{\textcolor[rgb]{0.13,0.29,0.53}{#1}}
\newcommand{\DecValTok}[1]{\textcolor[rgb]{0.00,0.00,0.81}{#1}}
\newcommand{\DocumentationTok}[1]{\textcolor[rgb]{0.56,0.35,0.01}{\textbf{\textit{#1}}}}
\newcommand{\ErrorTok}[1]{\textcolor[rgb]{0.64,0.00,0.00}{\textbf{#1}}}
\newcommand{\ExtensionTok}[1]{#1}
\newcommand{\FloatTok}[1]{\textcolor[rgb]{0.00,0.00,0.81}{#1}}
\newcommand{\FunctionTok}[1]{\textcolor[rgb]{0.13,0.29,0.53}{\textbf{#1}}}
\newcommand{\ImportTok}[1]{#1}
\newcommand{\InformationTok}[1]{\textcolor[rgb]{0.56,0.35,0.01}{\textbf{\textit{#1}}}}
\newcommand{\KeywordTok}[1]{\textcolor[rgb]{0.13,0.29,0.53}{\textbf{#1}}}
\newcommand{\NormalTok}[1]{#1}
\newcommand{\OperatorTok}[1]{\textcolor[rgb]{0.81,0.36,0.00}{\textbf{#1}}}
\newcommand{\OtherTok}[1]{\textcolor[rgb]{0.56,0.35,0.01}{#1}}
\newcommand{\PreprocessorTok}[1]{\textcolor[rgb]{0.56,0.35,0.01}{\textit{#1}}}
\newcommand{\RegionMarkerTok}[1]{#1}
\newcommand{\SpecialCharTok}[1]{\textcolor[rgb]{0.81,0.36,0.00}{\textbf{#1}}}
\newcommand{\SpecialStringTok}[1]{\textcolor[rgb]{0.31,0.60,0.02}{#1}}
\newcommand{\StringTok}[1]{\textcolor[rgb]{0.31,0.60,0.02}{#1}}
\newcommand{\VariableTok}[1]{\textcolor[rgb]{0.00,0.00,0.00}{#1}}
\newcommand{\VerbatimStringTok}[1]{\textcolor[rgb]{0.31,0.60,0.02}{#1}}
\newcommand{\WarningTok}[1]{\textcolor[rgb]{0.56,0.35,0.01}{\textbf{\textit{#1}}}}
\usepackage{graphicx}
\makeatletter
\newsavebox\pandoc@box
\newcommand*\pandocbounded[1]{% scales image to fit in text height/width
  \sbox\pandoc@box{#1}%
  \Gscale@div\@tempa{\textheight}{\dimexpr\ht\pandoc@box+\dp\pandoc@box\relax}%
  \Gscale@div\@tempb{\linewidth}{\wd\pandoc@box}%
  \ifdim\@tempb\p@<\@tempa\p@\let\@tempa\@tempb\fi% select the smaller of both
  \ifdim\@tempa\p@<\p@\scalebox{\@tempa}{\usebox\pandoc@box}%
  \else\usebox{\pandoc@box}%
  \fi%
}
% Set default figure placement to htbp
\def\fps@figure{htbp}
\makeatother
\setlength{\emergencystretch}{3em} % prevent overfull lines
\providecommand{\tightlist}{%
  \setlength{\itemsep}{0pt}\setlength{\parskip}{0pt}}
\setcounter{secnumdepth}{-\maxdimen} % remove section numbering
\usepackage{bookmark}
\IfFileExists{xurl.sty}{\usepackage{xurl}}{} % add URL line breaks if available
\urlstyle{same}
\hypersetup{
  pdftitle={In host viral load model},
  pdfauthor={Rob Challen},
  hidelinks,
  pdfcreator={LaTeX via pandoc}}

\title{In host viral load model}
\author{Rob Challen}
\date{2025-08-28}

\begin{document}
\maketitle

\section{Model desiderata}\label{model-desiderata}

\begin{itemize}
\tightlist
\item
  Sensible viral peak in naive host that can be translated into a viral
  load PCR measurement, with fairly predictable peak of virus, so we can
  determine a test signal / noise parameter for sensitivity and
  specificity, and make the PCR translation sensible.
\item
  Viral peak related to infectiousness, and producing a typical latent
  period / infectious period / delay distribution for secondary
  infections.
\item
  Tunable peak viral load and viral shedding duration.
\item
  Some measure of severity that allows us to express risk of illness
  (e.g.~symptoms, hospitalisation, death) in terms of modelled
  quantities.
\item
  Tunable host response to infection leading to different severity for
  given viral challenge
\item
  Ability to vaccinate individuals leading to less severe or lower viral
  peaks.
\item
  Waning of immunity. Tunable.
\item
  Secondary milder disease course, on second viral challenge.
\item
  Complete elimination of infection possible (or stable very low level
  of virus).
\item
  Chronic infection state possible.
\end{itemize}

\section{Compartments}\label{compartments}

\(V\) is count of virions.

\(T\) is total target cells, \(S\), susceptible targets, \(E\) is
``exposed'' targets (infected but not yet producing virus), \(I\) is
infected ( and producing virus), \(R\), ``removed'' are target cells
inactivated by immunity.

\(J\) is the total number of immune cells, of which \(D\) are dormant,
\(P\) are priming, and \(A\) are active.

\(\alpha\) is the fraction of immune to target cells. This affects how
effective the immune system is at clearing the virus.

\[
\begin{align}
V \\
1 &= S + E + I + R\\
\alpha &= D + P + A\\
\end{align}
\]

\subsection{Virions}\label{virions}

\(X_t\) is the time varying external viral exposure. \(\beta_{neut}\) is
a rate at which virions are neutralised by the immune system, after
which they are removed from the system. \(\beta_{inf}\) is a rate at
which virions infect susceptible target cells. \(\beta_{rep}\) is a rate
at which infected target cell replicate virus.

A continuous deterministic model might be:

\$\$ \begin{align}

V_{add} &= \beta_{rep} \times I \\
\beta_{remov} &= \beta_{neut} \times A + \beta_{inf} \times S \\ 
V_{remov} &= \beta_{remov} \times V \\
V_{neut} &= \frac{\beta_{neut} \times A}{\beta_{remov}} \times V_{remov} \\
V_{inf} &= \frac{\beta_{inf} \times S}{\beta_{remov}} \times V_{remov}\\

\frac{dV}{dt} &= X_t - V_{remov} + V_{add}\\

\end{align} \$\$

\subsection{Targets}\label{targets}

The interaction between virions and susceptible target cells has to take
into account the possibility of multiple virions infecting the same
cell.

This is different in the stochastic versus the continuous case, as in
the former we can use the absolute counts to get a precise value for the
probability of virions interacting with multiple.

\(\beta_{recov}\) is a recovery rate of target cells, \(\beta_{EI}\) is
the rate ``exposed'' targets convert to ``infected'' cells which produce
new virions. \(\gamma\) is a relative probability of discovery of
exposed cells compared to infected.

TODO: think about moving innoculation to S-\textgreater E transition
rather than adding to viral component. Would need to scale innoculation
by the immune activity, before the \(\beta_{infcell}\) to get similar
mucosal immune effect.

\[
\begin{align}
\beta_{infcell} &= \beta_{inf} \times \bigg(1-e^{-\frac{V_{inf}}{S}}\bigg) \\
S_E &= \beta_{infcell} \times S \\
\beta_{E-} &= \gamma\beta_{neut} \times A + \beta_{EI} \\
E_{-} &= \beta_{E-} \times E \\
E_R &= \frac{\gamma\beta_{neut} \times A}{\beta_{E-}} \times E_{-} \\
E_I &= \frac{\beta_{EI}}{\beta_{E-}} \times E_{-} \\
I_R &= \beta_{neut} \times A \times I \\
\frac{dS}{dt} &= \beta_{recov} R - S_E \\
\frac{dE}{dt} &= S_E - E_{-} \\
\frac{dI}{dt} &= E_I - I_R\\
\frac{dR}{dt} &= E_R + I_R - \beta_{recov} R\\ 
\end{align}
\]

\subsection{Immunity}\label{immunity}

\(V_t\) is a time varying vaccination dose. \(\beta_{induce}\) is the
rate at which exposed and infected targets. \(\beta_{PA}\) is the rate
at which primed immune cells transition to active. \(\beta_{waning}\) is
the rate at which active immune cells become dormant.

The SEIRS target cell model, with phases susceptible, exposed, producing
infective virus, regenerating is coupled to a SEIS-like model for
immunity where the phases as dormant, priming, active, dormant.

\[
\begin{align}
D_P &= \beta_{induce} \times \alpha(E+I) \times D \\
\frac{dD}{dt} &= \beta_{waning} A - D_P \\
\frac{dP}{dt} &= D_P - \beta_{PA} P \\
\frac{dA}{dt} &= \beta_{PA} P - \beta_{waning} A\\
\end{align}
\]

\section{model implementation}\label{model-implementation}

\begin{Shaded}
\begin{Highlighting}[]
\NormalTok{ode }\OtherTok{=} \CommentTok{\#odin.dust::odin\_dust(\{}
\NormalTok{  odin}\SpecialCharTok{::}\FunctionTok{odin}\NormalTok{(\{}
  
  \CommentTok{\# Initial state}
  
\NormalTok{  ratio\_immune\_target }\OtherTok{=} \FunctionTok{user}\NormalTok{(}\DecValTok{1}\NormalTok{)}
  
  \FunctionTok{initial}\NormalTok{(virions) }\OtherTok{=} \DecValTok{0}
  
  \FunctionTok{initial}\NormalTok{(target\_susceptible) }\OtherTok{=} \DecValTok{1}
  \FunctionTok{initial}\NormalTok{(target\_exposed) }\OtherTok{=} \DecValTok{0}
  \FunctionTok{initial}\NormalTok{(target\_infected) }\OtherTok{=} \DecValTok{0}
  
\NormalTok{  target\_removed }\OtherTok{=} \DecValTok{1} \SpecialCharTok{{-}}\NormalTok{ (target\_susceptible }\SpecialCharTok{+}\NormalTok{ target\_exposed }\SpecialCharTok{+}\NormalTok{ target\_infected)}
  
  \FunctionTok{initial}\NormalTok{(immune\_priming) }\OtherTok{=} \DecValTok{0}
  \FunctionTok{initial}\NormalTok{(immune\_active) }\OtherTok{=} \DecValTok{0}
\NormalTok{  immune\_dormant }\OtherTok{=}\NormalTok{ ratio\_immune\_target }\SpecialCharTok{{-}}\NormalTok{ immune\_priming }\SpecialCharTok{{-}}\NormalTok{ immune\_active}
  
  \CommentTok{\# Virions}
  
  \FunctionTok{deriv}\NormalTok{(virions) }\OtherTok{=}\NormalTok{ virions\_added }\SpecialCharTok{{-}}\NormalTok{ virions\_removed}
  
\NormalTok{  rate\_infection }\OtherTok{=} \FunctionTok{user}\NormalTok{(}\DecValTok{1}\NormalTok{)}
\NormalTok{  rate\_virion\_replication }\OtherTok{=} \FunctionTok{user}\NormalTok{(}\DecValTok{2}\NormalTok{)}
\NormalTok{  rate\_neutralization }\OtherTok{=}\NormalTok{ rate\_virion\_replication}
  
\NormalTok{  virions\_added }\OtherTok{=}\NormalTok{ rate\_virion\_replication }\SpecialCharTok{*}\NormalTok{ target\_infected}
  
\NormalTok{  rate\_virion\_removal }\OtherTok{=}\NormalTok{ ( rate\_neutralization }\SpecialCharTok{*}\NormalTok{ immune\_active }\SpecialCharTok{+}
\NormalTok{      rate\_infection }\SpecialCharTok{*}\NormalTok{ target\_susceptible )}
  
\NormalTok{  virions\_removed }\OtherTok{=}\NormalTok{ virions }\SpecialCharTok{*}\NormalTok{ rate\_virion\_removal}
\NormalTok{  virions\_neutralized }\OtherTok{=}\NormalTok{ virions\_removed }\SpecialCharTok{*}\NormalTok{ (rate\_neutralization }\SpecialCharTok{*}\NormalTok{ immune\_active)}\SpecialCharTok{/}\NormalTok{rate\_virion\_removal}
\NormalTok{  virions\_infecting }\OtherTok{=}\NormalTok{ virions\_removed }\SpecialCharTok{*}\NormalTok{ (rate\_infection }\SpecialCharTok{*}\NormalTok{ target\_susceptible)}\SpecialCharTok{/}\NormalTok{rate\_virion\_removal}
  
  
  
  \CommentTok{\# Targets}
  
  \FunctionTok{deriv}\NormalTok{(target\_susceptible) }\OtherTok{=}\NormalTok{ target\_recovered }\SpecialCharTok{{-}}\NormalTok{ target\_newly\_exposed}
  \FunctionTok{deriv}\NormalTok{(target\_exposed) }\OtherTok{=}\NormalTok{ target\_newly\_exposed }\SpecialCharTok{{-}}\NormalTok{ target\_exposed\_removed}
  \FunctionTok{deriv}\NormalTok{(target\_infected) }\OtherTok{=}\NormalTok{ target\_start\_infected }\SpecialCharTok{{-}}\NormalTok{ target\_infected\_removed}
  
\NormalTok{  rate\_target\_recovery }\OtherTok{=} \FunctionTok{user}\NormalTok{(}\DecValTok{1}\SpecialCharTok{/}\DecValTok{7}\NormalTok{)}
\NormalTok{  rate\_infected\_given\_exposed }\OtherTok{=}\NormalTok{ rate\_virion\_replication}
\NormalTok{  rate\_cellular\_removal }\OtherTok{=}\NormalTok{ rate\_neutralization}
  
  
  \CommentTok{\# https://math.stackexchange.com/questions/32800/probability{-}distribution{-}of{-}coverage{-}of{-}a{-}set{-}after{-}x{-}independently{-}randomly/32816\#32816 in the limit where N is large}
\NormalTok{  target\_newly\_exposed }\OtherTok{=}\NormalTok{ rate\_infection }\SpecialCharTok{*}\NormalTok{ target\_susceptible }\SpecialCharTok{*}\NormalTok{ (}\DecValTok{1}\SpecialCharTok{{-}}\FunctionTok{exp}\NormalTok{(}\SpecialCharTok{{-}}\NormalTok{virions\_infecting}\SpecialCharTok{/}\NormalTok{target\_susceptible))}

\NormalTok{  target\_recovered }\OtherTok{=}\NormalTok{ target\_removed }\SpecialCharTok{*}\NormalTok{ rate\_target\_recovery}
  
\NormalTok{  p\_propensity\_chronic }\OtherTok{=} \FunctionTok{user}\NormalTok{(}\DecValTok{0}\NormalTok{)}
  
\NormalTok{  rate\_target\_exposed\_removed }\OtherTok{=}\NormalTok{ (}\DecValTok{1}\SpecialCharTok{{-}}\NormalTok{p\_propensity\_chronic) }\SpecialCharTok{*}\NormalTok{ rate\_cellular\_removal }\SpecialCharTok{*}\NormalTok{ immune\_active }\SpecialCharTok{+}
\NormalTok{    rate\_infected\_given\_exposed}
\NormalTok{  target\_exposed\_removed }\OtherTok{=}\NormalTok{ target\_exposed }\SpecialCharTok{*}\NormalTok{ rate\_target\_exposed\_removed}
  
\NormalTok{  target\_exposed\_neut }\OtherTok{=}\NormalTok{ target\_exposed\_removed }\SpecialCharTok{*}\NormalTok{ (}\DecValTok{1}\SpecialCharTok{{-}}\NormalTok{p\_propensity\_chronic) }\SpecialCharTok{*}\NormalTok{ rate\_cellular\_removal }\SpecialCharTok{*}\NormalTok{ immune\_active }\SpecialCharTok{/}\NormalTok{  rate\_target\_exposed\_removed}
\NormalTok{  target\_start\_infected }\OtherTok{=}\NormalTok{ target\_exposed\_removed }\SpecialCharTok{*}\NormalTok{ rate\_infected\_given\_exposed }\SpecialCharTok{/}\NormalTok{ rate\_target\_exposed\_removed}
  
\NormalTok{  target\_infected\_removed }\OtherTok{=}\NormalTok{ target\_infected }\SpecialCharTok{*}\NormalTok{ rate\_cellular\_removal }\SpecialCharTok{*}\NormalTok{ immune\_active}
  
  
  \CommentTok{\# Immunity}
  
  
  
\NormalTok{  rate\_priming\_given\_infected }\OtherTok{=} \FunctionTok{user}\NormalTok{(}\DecValTok{1}\NormalTok{)}
\NormalTok{  rate\_active\_given\_priming }\OtherTok{=}\NormalTok{ rate\_priming\_given\_infected}
\NormalTok{  rate\_senescence\_given\_active }\OtherTok{=} \FunctionTok{user}\NormalTok{(}\DecValTok{1}\SpecialCharTok{/}\DecValTok{150}\NormalTok{)}
  
  
\NormalTok{  immune\_start\_priming }\OtherTok{=}\NormalTok{ immune\_dormant }\SpecialCharTok{*}\NormalTok{ rate\_priming\_given\_infected }\SpecialCharTok{*}\NormalTok{ (target\_exposed}\SpecialCharTok{+}\NormalTok{target\_infected)}
  
\NormalTok{  immune\_start\_active }\OtherTok{=}\NormalTok{ immune\_priming }\SpecialCharTok{*}\NormalTok{ rate\_active\_given\_priming}

  \FunctionTok{deriv}\NormalTok{(immune\_priming) }\OtherTok{=}\NormalTok{ immune\_start\_priming }\SpecialCharTok{{-}}\NormalTok{ immune\_start\_active}
      
\NormalTok{  immune\_senescence }\OtherTok{=}\NormalTok{ immune\_active }\SpecialCharTok{*}\NormalTok{ rate\_senescence\_given\_active}
    
  \FunctionTok{deriv}\NormalTok{(immune\_active) }\OtherTok{=}\NormalTok{ immune\_start\_active }\SpecialCharTok{{-}}\NormalTok{ immune\_senescence}
  
  
  
\NormalTok{\}, }\AttributeTok{skip\_cache =} \ConstantTok{TRUE}\NormalTok{)}
\end{Highlighting}
\end{Shaded}

\begin{verbatim}
## Loading required namespace: pkgbuild
\end{verbatim}

\begin{verbatim}
## Unused equations: target_exposed_neut, virions_neutralized
##  target_exposed_neut = target_exposed_removed * (1 - p_propensity_chronic) * 
##     rate_cellular_removal * immune_active/rate_target_exposed_removed # (line 30)
##  virions_neutralized = virions_removed * (rate_neutralization * 
##     immune_active)/rate_virion_removal # (line 17)
\end{verbatim}

\begin{verbatim}
## Generating model in c
\end{verbatim}

\begin{verbatim}
## i Re-compiling odinb3e2858c (debug build)
\end{verbatim}

\begin{verbatim}
## -- R CMD INSTALL ---------------------------------------------------------------
##      Loading required namespace: usethis
##      Loading required namespace: tidyr
##      Loading required namespace: systemfonts
##      Loading required namespace: svglite
##      Loading required namespace: stringr
##      Loading required namespace: scales
##      Loading required namespace: rsvg
##      Loading required namespace: rmarkdown
##      Loading required namespace: readr
##      Loading required namespace: rappdirs
##      Loading required namespace: ragg
##      Loading required namespace: pdftools
##      Loading required namespace: officer
##      Loading required namespace: huxtable
##      Loading required namespace: ggplot2
##      Loading required namespace: flextable
##      Loading required namespace: base64enc
##   -  installing *source* package ‘odinb3e2858c’ ...
##    ** this is package ‘odinb3e2858c’ version ‘0.0.1’
##    ** using staged installation
##      ** libs
##      using C compiler: ‘gcc (Ubuntu 11.4.0-1ubuntu1~22.04.2) 11.4.0’
##      gcc -I"/usr/share/R/include" -DNDEBUG       -fpic  -g -O2 -ffile-prefix-map=/build/r-base-0RQCNp/r-base-4.5.1=. -fstack-protector-strong -Wformat -Werror=format-security -Wdate-time -D_FORTIFY_SOURCE=2  -UNDEBUG -Wall -pedantic -g -O0 -c odin.c -o odin.o
##      odin.c: In function ‘odin_metadata’:
##    odin.c:159:18: warning: unused variable ‘internal’ [-Wunused-variable]
##      159 |   odin_internal *internal = odin_get_internal(internal_p, 1);
##          |                  ^~~~~~~~
##      gcc -I"/usr/share/R/include" -DNDEBUG       -fpic  -g -O2 -ffile-prefix-map=/build/r-base-0RQCNp/r-base-4.5.1=. -fstack-protector-strong -Wformat -Werror=format-security -Wdate-time -D_FORTIFY_SOURCE=2  -UNDEBUG -Wall -pedantic -g -O0 -c registration.c -o registration.o
##      gcc -shared -L/usr/lib/R/lib -Wl,-Bsymbolic-functions -flto=auto -ffat-lto-objects -flto=auto -Wl,-z,relro -o odinb3e2858c.so odin.o registration.o -L/usr/lib/R/lib -lR
##      installing to /tmp/Rtmp5fggMt/devtools_install_6170474a8d97/00LOCK-file61703afc9e49/00new/odinb3e2858c/libs
##      ** checking absolute paths in shared objects and dynamic libraries
##   -  DONE (odinb3e2858c)
## 
\end{verbatim}

\begin{verbatim}
## i Loading odinb3e2858c
\end{verbatim}

\subsection{Host factors}\label{host-factors}

\begin{Shaded}
\begin{Highlighting}[]
\NormalTok{tmp }\OtherTok{=} \FunctionTok{sweep}\NormalTok{(}\AttributeTok{parsDf =} \FunctionTok{tibble}\NormalTok{(}
  
  \AttributeTok{ratio\_immune\_target =} \FunctionTok{c}\NormalTok{(}\FloatTok{0.25}\NormalTok{,}\FloatTok{0.5}\NormalTok{,}\DecValTok{1}\NormalTok{,}\DecValTok{2}\NormalTok{,}\DecValTok{4}\NormalTok{), }\CommentTok{\# \textless{}\textless{} HOST FACTOR, duration and severity of disease, immune activity \& peak of virions, can lead to chronic carrier state}
  
  \AttributeTok{p\_propensity\_chronic =} \DecValTok{0}\NormalTok{, }\CommentTok{\#c(0,0.25, 0.5,0.75,1), \# \textless{}\textless{} HOST PROB CHRONIC}
  
  \AttributeTok{rate\_priming\_given\_infected =} \DecValTok{1}\NormalTok{, }\CommentTok{\#seq(0.7,1.3,0.1), \# c(0.25,0.5,1,2,4,8), \# severity roughly negatively linear with this. \textless{}\textless{} HOST FACTOR, duration, severity, and peak of virus, immunity fixed and no chrnoic disease effect}
  
  \AttributeTok{rate\_senescence\_given\_active =} \DecValTok{1}\SpecialCharTok{/}\DecValTok{150}\NormalTok{, }\CommentTok{\#1/800,}
  \AttributeTok{rate\_target\_recovery =} \DecValTok{1}\SpecialCharTok{/}\DecValTok{7}\NormalTok{, }\CommentTok{\#1/c(1,3,7,15,30), \# \textless{}\textless{} HOST FACTOR, duration and severity of disease, no effect on peak viral load ? useful for asymptomatics}
\NormalTok{))}

\NormalTok{tmp }\SpecialCharTok{\%\textgreater{}\%} \FunctionTok{plot\_sweep}\NormalTok{(}\SpecialCharTok{\textasciitilde{}}\NormalTok{variable, }\StringTok{"ratio immune to target"}\NormalTok{)}
\end{Highlighting}
\end{Shaded}

\begin{verbatim}
## `summarise()` has grouped output by 'ratio_immune_target',
## 'p_propensity_chronic', 'rate_priming_given_infected',
## 'rate_senescence_given_active', 'rate_target_recovery', 'set', 'step'. You can
## override using the `.groups` argument.
\end{verbatim}

\begin{verbatim}
## [[1]]
\end{verbatim}

\pandocbounded{\includegraphics[keepaspectratio]{in-host-ode_files/figure-latex/unnamed-chunk-4-1.pdf}}

\begin{verbatim}
## 
## [[2]]
\end{verbatim}

\pandocbounded{\includegraphics[keepaspectratio]{in-host-ode_files/figure-latex/unnamed-chunk-4-2.pdf}}

\subsection{Viral factors}\label{viral-factors}

rate of virion replication defines how many viral copies are produced
per infected cells and also how quickly exposed cells transition to
infected. It is a measure of how effiencetly the virus replicates once
in the cell

\begin{Shaded}
\begin{Highlighting}[]
\NormalTok{tmp }\OtherTok{=} \FunctionTok{sweep}\NormalTok{(}\AttributeTok{parsDf =} \FunctionTok{tibble}\NormalTok{(}
  
  \CommentTok{\# shortens GT? worsens severity}
  \AttributeTok{rate\_virion\_replication =} \DecValTok{4}\NormalTok{, }\CommentTok{\#c(1,2,4,8,16,32), \#1, \#c(1.5,2,2.5), \#c(0.5,1,2,4), \# c(0.5,1,2,4,8), \textless{}\textless{} VIRAL FACTOR}
  \CommentTok{\# shorten GT worsen severity ?higher values lead to chronic disease, if exposed are not cleared as more cellular reservoir?}
  \AttributeTok{rate\_infection =} \FunctionTok{c}\NormalTok{(}\FloatTok{0.25}\NormalTok{,}\FloatTok{0.5}\NormalTok{,}\DecValTok{1}\NormalTok{,}\DecValTok{2}\NormalTok{,}\DecValTok{4}\NormalTok{,}\DecValTok{8}\NormalTok{) }\CommentTok{\#1 \#rate\_virion\_replication, \#1, \#c(0.8, 0.9, 1, 1.1, 1.2, 2), \#c(3,2,1,0.5) \# 1/rate\_virion\_replication * 2, \#c(0.25,0.5,1,2,4,8), \# * seq(1,2,0.1),}
  
\NormalTok{))}

\NormalTok{tmp }\SpecialCharTok{\%\textgreater{}\%} \FunctionTok{plot\_sweep}\NormalTok{(}\SpecialCharTok{\textasciitilde{}}\NormalTok{variable, }\StringTok{"infection rate"}\NormalTok{)}
\end{Highlighting}
\end{Shaded}

\begin{verbatim}
## `summarise()` has grouped output by 'rate_virion_replication',
## 'rate_infection', 'set', 'step'. You can override using the `.groups` argument.
\end{verbatim}

\begin{verbatim}
## [[1]]
\end{verbatim}

\pandocbounded{\includegraphics[keepaspectratio]{in-host-ode_files/figure-latex/unnamed-chunk-5-1.pdf}}

\begin{verbatim}
## 
## [[2]]
\end{verbatim}

\pandocbounded{\includegraphics[keepaspectratio]{in-host-ode_files/figure-latex/unnamed-chunk-5-2.pdf}}

\begin{Shaded}
\begin{Highlighting}[]
\NormalTok{tmp2 }\OtherTok{=} \FunctionTok{sweep}\NormalTok{(}\AttributeTok{parsDf =} \FunctionTok{tibble}\NormalTok{(}
  
  \CommentTok{\# shortens GT? worsens severity}
  \AttributeTok{rate\_virion\_replication =} \FunctionTok{c}\NormalTok{(}\DecValTok{1}\NormalTok{,}\DecValTok{2}\NormalTok{,}\DecValTok{4}\NormalTok{,}\DecValTok{8}\NormalTok{,}\DecValTok{16}\NormalTok{,}\DecValTok{32}\NormalTok{), }\CommentTok{\#1, \#c(1.5,2,2.5), \#c(0.5,1,2,4), \# c(0.5,1,2,4,8), \textless{}\textless{} VIRAL FACTOR}
  \CommentTok{\# shorten GT worsen severity ?higher values lead to chronic disease, if exposed are not cleared as more cellular reservoir?}
  \AttributeTok{rate\_infection =} \DecValTok{4}\SpecialCharTok{/}\NormalTok{rate\_virion\_replication, }\CommentTok{\#rev(c(0.25,0.5,1,2,4,8)) \#1 \#rate\_virion\_replication, \#1, \#c(0.8, 0.9, 1, 1.1, 1.2, 2), \#c(3,2,1,0.5) \# 1/rate\_virion\_replication * 2, \#c(0.25,0.5,1,2,4,8), \# * seq(1,2,0.1),}
  
\NormalTok{))}

\NormalTok{tmp2 }\SpecialCharTok{\%\textgreater{}\%} \FunctionTok{plot\_sweep}\NormalTok{(}\SpecialCharTok{\textasciitilde{}}\NormalTok{variable, }\StringTok{"virion replication rate"}\NormalTok{)}
\end{Highlighting}
\end{Shaded}

\begin{verbatim}
## `summarise()` has grouped output by 'rate_virion_replication',
## 'rate_infection', 'set', 'step'. You can override using the `.groups` argument.
\end{verbatim}

\begin{verbatim}
## [[1]]
\end{verbatim}

\pandocbounded{\includegraphics[keepaspectratio]{in-host-ode_files/figure-latex/unnamed-chunk-6-1.pdf}}

\begin{verbatim}
## 
## [[2]]
\end{verbatim}

\pandocbounded{\includegraphics[keepaspectratio]{in-host-ode_files/figure-latex/unnamed-chunk-6-2.pdf}}

The rate of infection defines how quickly virions enter cells. Things
like spike protein mutations are likely to change this and this is the
physiological candidate for a more infectious variant (although reduced
efficiency of clearance could also a factor here).

\begin{Shaded}
\begin{Highlighting}[]
\NormalTok{tmp }\OtherTok{=} \FunctionTok{sweep}\NormalTok{(}\AttributeTok{parsDf =} \FunctionTok{tibble}\NormalTok{(}
  
  \CommentTok{\# shortens GT? worsens severity}
  \AttributeTok{rate\_virion\_replication =} \DecValTok{4}\NormalTok{, }\CommentTok{\#c(1,2,4,8,16,32), \#1, \#c(1.5,2,2.5), \#c(0.5,1,2,4), \# c(0.5,1,2,4,8), \textless{}\textless{} VIRAL FACTOR}
  \CommentTok{\# shorten GT worsen severity ?higher values lead to chronic disease, if exposed are not cleared as more cellular reservoir?}
  \AttributeTok{rate\_infection =} \FunctionTok{c}\NormalTok{(}\FloatTok{0.25}\NormalTok{,}\FloatTok{0.5}\NormalTok{,}\DecValTok{1}\NormalTok{,}\DecValTok{2}\NormalTok{,}\DecValTok{4}\NormalTok{,}\DecValTok{8}\NormalTok{) }\CommentTok{\#1 \#rate\_virion\_replication, \#1, \#c(0.8, 0.9, 1, 1.1, 1.2, 2), \#c(3,2,1,0.5) \# 1/rate\_virion\_replication * 2, \#c(0.25,0.5,1,2,4,8), \# * seq(1,2,0.1),}
  
\NormalTok{))}

\NormalTok{tmp }\SpecialCharTok{\%\textgreater{}\%} \FunctionTok{plot\_sweep}\NormalTok{(}\SpecialCharTok{\textasciitilde{}}\NormalTok{variable, }\StringTok{"infection rate"}\NormalTok{)}
\end{Highlighting}
\end{Shaded}

\begin{verbatim}
## `summarise()` has grouped output by 'rate_virion_replication',
## 'rate_infection', 'set', 'step'. You can override using the `.groups` argument.
\end{verbatim}

\begin{verbatim}
## [[1]]
\end{verbatim}

\pandocbounded{\includegraphics[keepaspectratio]{in-host-ode_files/figure-latex/unnamed-chunk-7-1.pdf}}

\begin{verbatim}
## 
## [[2]]
\end{verbatim}

\pandocbounded{\includegraphics[keepaspectratio]{in-host-ode_files/figure-latex/unnamed-chunk-7-2.pdf}}

\subsection{Invasion and infection}\label{invasion-and-infection}

Infection of the organism can be thought of as the conditions underwhich
replication in host is established. This is analagous to R0 in a SEIR
model. At the disease free equilibrium point we consider the next
generation matrices of the infected classes E and I.

\[
\begin{align}
\beta_{infcell} &= \beta_{inf} \times \bigg(1-e^{-\frac{V_{inf}}{S}}\bigg) \\
S_E &= \beta_{infcell} \times S \\
\beta_{E-} &= \gamma\beta_{neut} \times A + \beta_{EI} \\
E_{-} &= \beta_{E-} \times E \\
E_R &= \frac{\gamma\beta_{neut} \times A}{\beta_{E-}} \times E_{-} \\
E_I &= \frac{\beta_{EI}}{\beta_{E-}} \times E_{-} \\
I_R &= \beta_{neut} \times A \times I \\
\frac{dS}{dt} &= \beta_{recov} R - S_E \\
\frac{dE}{dt} &= S_E - E_{-} \\
\frac{dI}{dt} &= E_I - I_R\\
\frac{dR}{dt} &= E_R + I_R - \beta_{recov} R\\ 
\end{align}
\]

linearize at disease free equilibrium, and assume known value for A

S = 1, E=I=R=0;

\[
\begin{align}
\dot{E} &= \beta_{infcell}I - (\gamma\beta_{neut} \times A + \beta_{EI}) \times E\\
\dot{I} &= \beta_{EI} \times E - \beta_{neut} \times A \times I \\
\end{align}
\]

\[
\begin{align}
F &= \begin{bmatrix}
0 & \beta_{infcell}S_0\\
0 & 0
\end{bmatrix} \\
V &= \begin{bmatrix}
- (\gamma\beta_{neut} \times A + \beta_{EI}) & 0 \\
\beta_{EI} & - \beta_{neut} \times A
\end{bmatrix} \\
\end{align}
\]

\[
\begin{align}
R_0 &= \frac{\beta_{EI}S_0\beta_{infcell}}{(\gamma\beta_{neut} \times A + \beta_{EI})(\beta_{neut} \times A)} \\
R_0 &= \frac{\beta_{EI}S_0\beta_{inf} \times \bigg(1-e^{-\frac{V_{inf}}{S_0}}\bigg)}{(\gamma\beta_{neut} \times A + \beta_{EI})(\beta_{neut} \times A)} \\
\end{align}
\]

\begin{Shaded}
\begin{Highlighting}[]
\NormalTok{mod }\OtherTok{=}\NormalTok{ ode}\SpecialCharTok{$}\FunctionTok{new}\NormalTok{()}
\NormalTok{tmp }\OtherTok{=}\NormalTok{ mod}\SpecialCharTok{$}\FunctionTok{run}\NormalTok{(}\FunctionTok{c}\NormalTok{(}\DecValTok{0}\NormalTok{,}\DecValTok{1}\NormalTok{))}
\NormalTok{conditions }\OtherTok{=}\NormalTok{ tmp[}\DecValTok{2}\NormalTok{,}\SpecialCharTok{{-}}\DecValTok{1}\NormalTok{]}

\NormalTok{R0 }\OtherTok{=} \ControlFlowTok{function}\NormalTok{(conditions, }\AttributeTok{params =}\NormalTok{ mod}\SpecialCharTok{$}\FunctionTok{contents}\NormalTok{()) \{}
\NormalTok{  tmp }\OtherTok{=} \FunctionTok{c}\NormalTok{(conditions,params)}
  \FunctionTok{with}\NormalTok{(tmp, \{}
\NormalTok{    (rate\_infected\_given\_exposed }\SpecialCharTok{*}\NormalTok{ target\_susceptible }\SpecialCharTok{*}\NormalTok{ rate\_infection }\SpecialCharTok{*}\NormalTok{ (}\DecValTok{1}\SpecialCharTok{{-}}\FunctionTok{exp}\NormalTok{(}\SpecialCharTok{{-}}\NormalTok{virions}\SpecialCharTok{/}\NormalTok{target\_susceptible))) }\SpecialCharTok{/} 
\NormalTok{      (((}\DecValTok{1}\SpecialCharTok{{-}}\NormalTok{p\_propensity\_chronic) }\SpecialCharTok{*}\NormalTok{ rate\_neutralization }\SpecialCharTok{*}\NormalTok{ immune\_active }\SpecialCharTok{+}\NormalTok{ rate\_infected\_given\_exposed) }\SpecialCharTok{*}\NormalTok{ (rate\_neutralization }\SpecialCharTok{*}\NormalTok{ immune\_active))}
\NormalTok{  \})}
\NormalTok{\}}



\NormalTok{conditions[[}\StringTok{"immune\_active"}\NormalTok{]] }\OtherTok{=} \FloatTok{0.7}
\NormalTok{conditions[[}\StringTok{"virions"}\NormalTok{]] }\OtherTok{=} \FloatTok{0.01}
\FunctionTok{R0}\NormalTok{(conditions)}
\end{Highlighting}
\end{Shaded}

\begin{verbatim}
## [1] 0.004180742
\end{verbatim}

\begin{Shaded}
\begin{Highlighting}[]
\NormalTok{y }\OtherTok{=}\NormalTok{ mod}\SpecialCharTok{$}\FunctionTok{run}\NormalTok{(}\FunctionTok{seq}\NormalTok{(}\DecValTok{0}\NormalTok{,}\DecValTok{100}\NormalTok{,}\FloatTok{0.1}\NormalTok{), }\AttributeTok{y=}\NormalTok{ conditions)}

\FunctionTok{plot}\NormalTok{(y)}
\end{Highlighting}
\end{Shaded}

\pandocbounded{\includegraphics[keepaspectratio]{in-host-ode_files/figure-latex/unnamed-chunk-8-1.pdf}}

\end{document}
